% DOCUMENT CLASS %
\documentclass[a4paper, 11pt]{article}   % for A4 Paper, 11pt fontsize and article spacing

% PACKAGES%
\usepackage{amsfonts}   % For a basic mathfont (many will be replaced by stix)
\usepackage{amsmath}    % For basic math symbols
\usepackage{bbm}        % For \mathbbm (better version of \mathbb)
\usepackage{mathrsfs}   % For \mathscr
\usepackage{hyperref}   % For footnotes
\usepackage{csquotes}   % For \textquote{}
\usepackage{IEEEtrantools}  % For better alignments
\usepackage{stix}   % For better mathfont
\usepackage{tikz}   % For a macro
\usepackage{listings} % For pasted code
\usepackage{paralist}

% SETS %
\newcommand*{\R}{\mathbbm R}    % Set of real numbers
\newcommand*{\N}{\mathbbm N}    % Set of natural numbers (beginning at 0)
\newcommand*{\Z}{\mathbbm Z}    % Set of integers
\newcommand*{\Q}{\mathbbm Q}    % Set of rational numbers
\newcommand*{\pro}{\mathcal P}  % Power set of a set
\newcommand*{\Co}{\mathcal C^1}     % Set of R^n -> R^d functions that are fully partially differenciable and continuous
\newcommand*{\Ct}{\mathcal C^2}     % Set of R^n -> R functions that are C^1 functions and whose gradient is C^1 function

% OPERATIONS ON FUNCTIONS %
\newcommand*{\ddx}{\frac{\text d}{\text dx}}    % Derivative of a function
\newcommand*{\grad}{\text{grad}}    % Gradient of a function
\newcommand*{\J}{\textbf{J}}        % Jacobi-matrix of a function
\newcommand*{\He}{\textbf{H}}       % Hesse-matrix of a function
\newcommand*{\LMAX}{\text{\scshape Lmax}}   % Set of maxima of a function
\newcommand*{\LMIN}{\text{\scshape Lmin}}   % Set of minima of a function
\newcommand*{\img}{\text{img}}      % Image of a function
\newcommand*{\dom}{\text{dom}}      % Domain of a function

% DISTRIBUTIONS %
\newcommand*{\Unif}{\text{Unif}}    % Uniform distribution (discrete or continous)
\newcommand*{\Ber}{\text{Ber}}  % Bernoulli distribution
\newcommand*{\Bin}{\text{Bin}}  % Binomial distribution
\newcommand*{\Exp}{\text{Exp}}  % Exponential distribution
\newcommand*{\Geo}{\text{Geo}}  % Geometric distribution
\newcommand*{\Pois}{\text{Pois}}    % Poisson distribution
\newcommand*{\Norm}{\mathcal{N}}    % Normal distribution
\newcommand*{\D}{\mathcal{D}}   % Some random distribution

% BASIC PROBABILISTIC STUFF %
\newcommand*{\E}{\mathcal E}    % Some space of events
\newcommand*{\Pro}{\mathbbm P}  % Probability function of an event
\newcommand*{\Od}{\text{Od}}    % Odds of an event

% OPERATIONS ON DISTRIBUTIONS OR RANDOM VALUES %
\newcommand*{\supp}{\text{supp}}    % Support of a random variable
\newcommand*{\Ew}{\mathbbm E}   % Expected value
\newcommand*{\Var}{\text{Var}}  % Variance
\newcommand*{\Cov}{\text{Cov}}  % Covariance of two random variables
\newcommand*{\B}{\mathbbm B}    % Bias of a point estimation

% MACROS %
\newcommand{\subsubsubsection}[2]{\textsc{\underbar{#1}} \\#2\\\\}  % Macro for title in ALLCAPS, ends with two newlines
\newcommand*{\todo}{\textit{\dots TODO \dots}}  % Macro for ... todo ...
\newcommand*{\QDp}{{:}\text{ }} % Macro for ": " so that when writing something like "\forall x:" there is not a seperation between the '\forall' and the ':'
\newcommand*{\puffer}{\text{ }\text{ }\text{ }\text{ }} % Macro for a lazy aligment 1
\newcommand*{\gedanke}{\textbf{-- }}    % Macro for a long minus followed by text
\newcommand*{\smi}{\text{-}}        % Macro for a small minus ('-')
\newcommand*{\gap}{\text{ }}    % Macro for a lazy aligment 2
\newcommand*{\vor}{\textbf{Vor.:} \gap} % Vor.
\newcommand*{\beh}{\textbf{Beh.:} \gap} % Beh.
\newcommand*{\bew}{\textbf{Bew.:} \gap} % Bew.
\newcommand*{\qed}{\null\nobreak\hfill\ensuremath{\square}} % Macro for a QED-Box

\begin{document}

\subsubsection*{Which file determines how to build a Docker image?}
Respective dockerfiles.
\subsubsection*{What is the difference between a Docker image and a container?}
A docker image determines what the service is, what languages and services 
it is running and so on. The container is the actual environment the service or 
application will be running in, virtualizing the environment on the local mahine.
Why do you need to set the port when running the containers?
So that docker knows through which ports to connect to the service from outside 
the container itself (as our web browser for example does not run in the 
docker container and thus needs to have a metaphorical "door" it can enter through).
The same principle applies to services that are running in multiple docker containers, 
as is the case here. Both containers, frontend and server, are separate from each other, 
and the specified ports facilitate the ability to communicate with each other.

\subsubsection*{What did you need to change in order to start a second server?}
We had to change the \verb|SERVER_LIST| value defined in the frontend dockerfile to 
include the second server started at \verb|127.0.0:8002:80|. After saving the changes we 
then had to rebuilt the image and restart the containers to display both connected servers 
in the frontend.

\subsubsection*{Why do changes in the code are not directly reflected in the application?}
For each change in the code the docker image has to be rebuilt before restarting 
the container. 

\subsubsection*{Which REST URIs are defined in the server?}
\begin{compactitem}
    \item \verb|/entries| (POST)
    \item \verb|/entries| (GET)
    \item \verb|/entries/<entry_id>| (PUT)
    \item \verb|/entries/<entry_id>| (DELETE)
    \item \verb|/message| (POST)
\end{compactitem}

\end{document}